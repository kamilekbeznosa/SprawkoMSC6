\documentclass{article}
\usepackage{graphicx} % Required for inserting images
\usepackage[utf8]{inputenc}
\usepackage[T1]{fontenc}
\usepackage{karnaugh-map}
\usepackage[polish]{babel}
\usepackage{subcaption}
\usepackage{float}
\usepackage{enumitem}
\usepackage{url}
\usepackage{array}
\usepackage{indentfirst}
\usepackage{amsmath}
\usepackage{pgfplots}
\pgfplotsset{compat=1.17}
\usepackage{longtable}

\begin{document}

\begin{center}
	% Używamy @{} aby usunąć zewnętrzne marginesy
	% Używamy 'm' (z pakietu 'array') dla środkowania w pionie
	
	\begin{tabular}{@{}|m{0.65\textwidth}|m{0.3\textwidth}|@{}}
		\hline
		% --- PIERWSZY WIERSZ LOGICZNY (wg obrazka) ---
		% Komórka 1,1 (Lewa-góra)
		Wydział Informatyki Politechniki Białostockiej \newline
		Przedmiot: Modułowe systemy cyfrowe
		&
		% Komórka 1,2 (Prawa-góra)
		Data: 10.12.2025 \\ \hline
		
		% --- DRUGI WIERSZ LOGICZNY (wg obrazka) ---
		% Komórka 2,1 (Lewa-środek)
		Zajęcia nr 6 \newline
		Temat: Bloki sekwencyjne-liczniki \newline
		
		& % <-- Separator kolumn dla wiersza 2
		
		% Komórka 2,2 (Prawa-środek)
		
		\\
		
		% --- TRZECI WIERSZ LOGICZNY (wg obrazka) ---
		% Komórka 3,1 (Lewa-dół)
		Grupa: Lab 8 \newline
		Imię i nazwisko: \newline
		Kamil Kubajewski, Jakub Matusiewicz, Bartosz Orłowski
		
		& % <-- Separator kolumn dla wiersza 3
		
		% Komórka 3,2 (Prawa-dół)
		Prowadzący: \newline
		dr hab. inż. Sławomir Zieliński \\ \hline
	\end{tabular}
\end{center}

\section{Cel ćwiczeń}
Zapoznanie z sekwencyjnymi układami cyfrowymi.
\section{Podstawa teoretyczna}


\section{Przebieg ćwiczeń}
\subsection{Zadanie 1}
Zaprojektuj i zrealizuj dwójkę liczącą na przerzutniku typu D i JK wykorzystując moduł
laboratoryjny DB11. Zaobserwuj sygnały wyjściowe obu przerzutników.

\subsubsection{Realizacja na przerzutniku typu D}
Przerzutnik typu D przepisuje stan wejścia $D$ na wyjście $Q$ po wystąpieniu aktywnego zbocza zegara. Aby układ działał jako dwójka licząca, należy na wejście podać stan przeciwny do aktualnego. Równanie charakterystyczne licznika MOD 2 dla przerzutnika D przyjmuje postać:
\begin{equation}
    D = \overline{Q_n}
\end{equation}

Poniższa tabela przedstawia zmianę stanu licznika ($Q_n \rightarrow Q_{n+1}$) oraz wymagany stan wejścia $D$.

\begin{table}[h!]
    \centering
    \caption{Tabela przejść i wzbudzeń przerzutnika D}
    \begin{tabular}{|c|c|c|l|}
    \hline
    \textbf{Stan obecny ($Q_n$)} & \textbf{Stan następny ($Q_{n+1}$)} & \textbf{Wymagane $D$} & \textbf{Opis} \\ \hline
    0 & 1 & 1 & Zmiana $0 \rightarrow 1$ (bo $D = \overline{Q}$) \\ \hline
    1 & 0 & 0 & Zmiana $1 \rightarrow 0$ (bo $D = \overline{Q}$) \\ \hline
    \end{tabular}
\end{table}

\subsubsection{Realizacja na przerzutniku typu JK}
Przerzutnik JK jest przerzutnikiem uniwersalnym. Jego działanie zależy od konfiguracji wejść $J$ i $K$. Aby uzyskać zmianę stanu na przeciwny przy każdym takcie zegara (tryb \textbf{Toggle}), czyli $Q_{n+1} = \overline{Q_n}$, należy ustawić oba wejścia w stan wysoki:
\begin{equation}
    J = 1, \quad K = 1
\end{equation}

Tabela przedstawia wymagane stany na wejściach $J$ i $K$, aby uzyskać zmianę stanu wyjścia.

\begin{table}[h!]
    \centering
    \caption{Tabela przejść i wzbudzeń przerzutnika JK}
    \begin{tabular}{|c|c|c|c|c|}
    \hline
    \textbf{Stan ($Q_n$)} & \textbf{Stan nast. ($Q_{n+1}$)} & \textbf{Wymagane $J$} & \textbf{Wymagane $K$} & \textbf{Tryb} \\ \hline
    0 & 1 & 1 & X & Toggle \\ \hline
    1 & 0 & X & 1 & Toggle \\ \hline
    \end{tabular}
    \vspace{0.5em} \\
    \textit{(Gdzie X oznacza stan dowolny. Dla uproszczenia konstrukcji przyjęto $J=1$ i $K=1$ dla obu przypadków).}
\end{table}

\subsubsection{Zrealizowany układ}
Na podstawie powyższych analiz teoretycznych zaprojektowano i zmontowano układ. Schemat połączeń oraz wynik weryfikacji przedstawiono na poniższym rysunku.

\begin{figure}[h!]
    \centering
    \includegraphics[width=0.8\textwidth]{obwod1.PNG}
    \caption{Zrealizowany licznik modulo 2 na przerzutnikach}
    \label{fig:obwod1}
\end{figure}

\subsection{Zadanie 2}
Zaprojektować i wykonać 4-bitowy licznik liczący w górę i w dół (ang. Ripple counter)
wykorzystując moduł laboratoryjny DB14.


\subsubsection{Projekt połączeń – Licznik w górę (UP)}
W celu realizacji zliczania narastającego (0, 1, 2...), zaprojektowano układ, w którym wyjście proste $Q$ poprzedniego przerzutnika steruje wejściem zegarowym $C_P$ następnego przerzutnika.

Zrealizowane połączenia:
\begin{enumerate}
    \item Źródło zegara $\rightarrow$ wejście $C_P$ bitu $Q_0$.
    \item Wyjście $Q_0$ $\rightarrow$ wejście $C_P$ bitu $Q_1$.
    \item Wyjście $Q_1$ $\rightarrow$ wejście $C_P$ bitu $Q_2$.
    \item Wyjście $Q_2$ $\rightarrow$ wejście $C_P$ bitu $Q_3$.
\end{enumerate}

Uzyskane stany wyjść przedstawiono w tabeli poniżej.

\begin{table}[h!]
    \centering
    \caption{Zarejestrowana sekwencja liczenia w górę}
    \begin{tabular}{|c|c|c|c|c|c|}
    \hline
    \textbf{Takt} & \textbf{$Q_3$} & \textbf{$Q_2$} & \textbf{$Q_1$} & \textbf{$Q_0$} & \textbf{Stan (DEC)} \\ \hline
    0 & 0 & 0 & 0 & 0 & 0 \\ \hline
    1 & 0 & 0 & 0 & 1 & 1 \\ \hline
    2 & 0 & 0 & 1 & 0 & 2 \\ \hline
    ... & ... & ... & ... & ... & ... \\ \hline
    15 & 1 & 1 & 1 & 1 & 15 \\ \hline
    \end{tabular}
\end{table}

\subsubsection{Projekt połączeń – Licznik w dół (DOWN)}
W celu realizacji zliczania malejącego (15, 14, 13...), zmodyfikowano projekt, wykorzystując wyjścia zanegowane $\overline{Q}$ do sterowania kolejnymi stopniami licznika.

Zrealizowane połączenia:
\begin{enumerate}
    \item Źródło zegara $\rightarrow$ wejście $C_P$ bitu $Q_0$ (bez zmian).
    \item Wyjście $\overline{Q_0}$ $\rightarrow$ wejście $C_P$ bitu $Q_1$.
    \item Wyjście $\overline{Q_1}$ $\rightarrow$ wejście $C_P$ bitu $Q_2$.
    \item Wyjście $\overline{Q_2}$ $\rightarrow$ wejście $C_P$ bitu $Q_3$.
\end{enumerate}

\subsubsection{Realizacja w środowisku symulacyjnym}
Układ został zaprojektowany i uruchomiony w programie Multisim. Poprawność połączeń zweryfikowano poprzez obserwację stanów na wyświetlaczu szesnastkowym oraz diodach próbników logicznych.

\begin{figure}[h!]
    \centering
    \includegraphics[width=0.9\textwidth]{obwod2.PNG}
    \caption{Realizacja 4-bitowego licznika w Multisim}
    \label{fig:licznik_ripple}
\end{figure}

\subsection{Zadanie 3}
Przebadać działanie 8-bitowego licznika cyklicznego (ang. Ring Counter) wykorzystując
moduł laboratoryjny DB37.

Weryfikacja działania układu polegała na ręcznym sterowaniu sygnałem zegarowym za pomocą przycisku \textbf{S3}. Obserwacja diod LED (\textbf{L1}--\textbf{L8}) pozwoliła potwierdzić poprawność logiczną układu: każde wciśnięcie przycisku powodowało cykliczne przemieszczanie się stanu aktywnego (zapalona dioda) na kolejną pozycję. Po osiągnięciu ostatniego bitu, cykl rozpoczynał się od nowa.

Dodatkowo przetestowano funkcję resetowania za pomocą przycisku \textbf{S1}, którego wciśnięcie powodowało natychmiastowe wyzerowanie licznika (zgaszenie wszystkich diod).
\begin{figure}[h!]
    \centering
    \includegraphics[width=0.9\textwidth]{obwod3.PNG}
    \caption{Realizacja 8-bitowego licznika cyklicznego}
    \label{fig:licznik_ripple}
\end{figure}
\section{Wnioski}


\section{Literatura}
\renewcommand{\refname}{}
\begin{thebibliography}{9}

\end{thebibliography}

\section{Protokół}
\begin{figure}[h]
    \centering
    \includegraphics[width=0.8\textwidth]{protokol.PNG}
    \caption{Protokół}
    \label{fig:moj_obrazek}
\end{figure}


\end{document}